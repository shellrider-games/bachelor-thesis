\chapter{Method}
\label{ch:method}

\section*{Literature review}
I reviewed previous work, focusing on two areas. I explored already available
methods for creating animations from sketches by performing skeleton
classification and reviewed previous work dealing with the classification of
sketched objects.

\section*{Related work}
\textcite{eitz2012hdhso} collected a dataset of 20,000 sketches and divided them
into 250 categories of 80 images each. Humans recognized on average 73.1\% of 
these sketches correctly. This dataset is used in my work to train and validate
the classifier to choose which animation is the most appropriate to show.

\textcite{10.1145/3469877.3490565} proposes a pipeline to create rigged and
animated characters from a single image. Their solution aims for a holistic
approach, requiring no user intervention, to assist non-professional users in
creating animated characters. The proposed pipeline performs contour extraction
with salient object detection and extrudes a 3D mesh from geometry generated by
applying constrained Delaunay to the contours. Afterwards, a skeleton is
estimated using a mean curve method and an animation is transferred onto the
skeleton ==DESCRIBE HOW HERE MAYBE==. In my work, I want to follow a similar
philosophy of no user interaction and hope to improve the believability of the
animated results by not only classifying the skeleton type but also the subject
class of the input sketch.

\section*{Training classification models}
I used a subset of the dataset provided by \textcite{eitz2012hdhso} to
train my classification models. Only the classes "cat" and "dog" were taken as
training data for my models. To train and evaluate my models I used the
scikit-learn library introduced by \textcite{scikit-learn}.